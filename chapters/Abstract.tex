\chapter*{Abstract}

\section*{English Version}
This thesis addresses the challenges of data scarcity
and noisy datasets in machine learning, with a focus on
offline deep reinforcement learning. We introduce
two novel frameworks for counterfactual data generation:
WRe-CTDG and S-CTDG.
These methods aim to augment pre-collected datasets by generating
additional high-fidelity experiences that align with the environment's
underlying transition dynamics.
We evaluate our frameworks across various environments,
comparing their performance against non-augmented baselines.
Results demonstrate significant improvements in reinforcement
learning performance, particularly for S-CTDG in complex environments,
at the same time identifying important trade-offs regarding its applicability.
This work contributes to the integration of causal inference and
machine learning, offering new approaches to leverage causal relationships
in data augmentation for offline deep reinforcement learning.

\section*{Versione Italiana}

Questa tesi affronta le sfide della scarsità di dati
e dei dataset rumorosi nel machine learning,
con un'attenzione particolare all deep reinforcement learning offline.
Introduciamo
due nuovi framework per la generazione di dati counterfactual:
WRe-CTDG e S-CTDG.
Questi sistemi mirano ad aumentare i dataset raccolti a priori generando
esperienze aggiuntive ad alta fedeltà che si allineano alle dinamiche
di transizione dell'ambiente.
Abbiamo valutato i nostri framework in vari ambienti,
confrontando le loro prestazioni con quelle di base senza aumento dei campioni.
I risultati dimostrano miglioramenti significativi nelle prestazioni,
in particolare per S-CTDG in ambienti più complessi, indentificandone
allo stesso tempo importanti compromessi per quanto riguarda la sua applicabilità.
Questo lavoro contribuisce all'integrazione dell'inferenza
causale nel machine learning, offrendo nuovi approcci per sfruttare
le relazioni causali
nell'aumento dei dati per il deep reinforcement learning offline.