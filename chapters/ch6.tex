\section{Summary of Findings}

This thesis has explored novel approaches to address the challenges
of data scarcity and noisy datasets in machine learning, with a specific
focus on offline deep reinforcement learning scenarios.
We introduced two frameworks for counterfactual data generation:
the Wasserstein Reward-enhanced CounTerfactual Data Generation (WRe-CTDG)
and the Supervised CounTerfactual Data Generation (S-CTDG).

Our experimental results demonstrated the efficacy of these
frameworks across various environments and highlight the potential
of counterfactual data generation in enhancing the performance of
offline reinforcement learning algorithms
in across diverse tasks and systems.

\section{Implications and Contributions}

The findings of this research have several important implications:
\begin{itemize}
    \item \textbf{Improved Data Utilization}: Our frameworks demonstrate
    that it's possible to extract more value from existing datasets
    through sophisticated augmentation techniques, potentially 
    reducing the need for extensive data collection.
    \item \textbf{Enhanced Model Performance}: The significant improvements
    in reward values, especially in complex environments
    like Half Cheetah and Ant, suggest that our methods can lead
    to more robust and effective reinforcement learning models.
    \item \textbf{Bridging Causal Inference and Machine Learning}:
    By incorporating Structural Causal Models and
    Generative Adversarial Networks, our work contributes to the
    growing field of causal machine learning, offering new ways to
    leverage causal relationships in data augmentation.
    \item \textbf{Applicability to Diverse Domains}: The effectiveness of
    our methods across both robotic simulations and healthcare
    scenarios indicates their potential broad applicability.
\end{itemize}

\section{Limitations}

Despite the promising results, it's important to acknowledge the
limitations of our approach:
\begin{itemize}
    \item \textbf{Prior Knowledge Requirement}: S-CTDG, while
    highly effective, requires prior knowledge of the system
    noise and the ability to perform counterfactual actions
    during data collection. This may limit its applicability
    in some real-world scenarios.
    \item \textbf{Computational Complexity}: The generation of
    counterfactual data, especially in complex environments,
    can be computationally intensive.
    \item \textbf{Generalization to Other Domains}: While our
    methods showed success in the tested environments, their
    effectiveness in other domains or more complex real-world
    scenarios remains to be fully explored.
\end{itemize}

\section{Future Research Directions}

Based on our findings and identified limitations,
several avenues for future research emerge:
\begin{itemize}
    \item \textbf{Relaxing Prior Knowledge Requirements}: Investigating
    methods to reduce the dependence on detailed prior knowledge of
    system dynamics could broaden the applicability of S-CTDG.
    \item \textbf{Scaling to More Complex Environments}: Exploring
    the effectiveness of these frameworks in increasingly complex
    and realistic environments would be valuable.
    \item \textbf{Real-World Applications}: Applying these
    techniques to real-world offline reinforcement learning problems
    to validate their practical utility.
    \item \textbf{Exploring Advanced ML Architectures}:
    Investigating the integration of state-of-the-art machine
    learning frameworks and structures, such as transformers
    (introduced in \cite{vaswani2023attentionneed})
    and diffusion models
    (presented in \cite{sohldickstein2015deepunsupervisedlearningusing}),
    into our counterfactual data generation approaches,
    for instance:
    \begin{itemize}
        \item \textbf{Transformer}-based models, could improve the
        capture of long-range dependencies in sequential data,
        potentially leading to more coherent and realistic
        counterfactual trajectories.
        \item \textbf{Diffusion models} might offer a new approach
        to generating high-quality counterfactual images, particularly
        in visually complex environments like robotics simulations.
    \end{itemize}
\end{itemize}

In conclusion, this thesis has demonstrated the potential of causal
inference-based counterfactual data generation in addressing challenges
of data scarcity and noise in offline reinforcement learning.
While there are limitations to be addressed, the promising results
open up exciting possibilities for future research and applications in
this rapidly evolving field.